\documentclass[12pt]{exam}

\usepackage[spanish,mexico]{babel}
\usepackage[latin1]{inputenc}
\usepackage[T1]{fontenc}
\usepackage{amsmath,amsfonts}

\usepackage{fullpage}

\pagestyle{head}
\extraheadheight{.6in}
\pointname{ puntos}
\lhead{Matem�ticas Discretas, Maestr�a en Matem�tica Educativa, UAEH\\
Primer examen, 28 de agosto de 2013}

\begin{document}

\hbox to \textwidth{NOMBRE:\enspace\hrulefill}
\bigskip{}

Recuerda que debes explicar tus respuestas. Respuestas con poca o nula
argumentaci�n adecuada recibir�n poco o nada de cr�dito. Escoge 5
preguntas, marcando de manera clara las preguntas
seleccionadas. Tienes dos horas para resolverlo.

\begin{questions}
  \question �En cu�ntas sucesiones de ocho volados se obtienen
  exactamente cinco �guilas y tres soles?

  \question �En cu�ntas de las $6!$ ``palabras'' de 6 letras distintas
  que se pueden formar con las letras ABCDEF se tiene que la A aparece
  junto a la D?

  \question De un grupo donde hay 10 mujeres y 10 hombres, �de cu�ntas
  maneras se puede escoger un comit� de 4 personas, donde debe haber 2
  mujeres y 2 hombres?

  \question En una quiniela de Pron�sticos Deportivos hay 13 partidos,
  y de cada uno debe escogerse una opci�n de entre: local, empate,
  visitante. �De cu�ntas maneras puede llenarse la quiniela?

  \question �Cu�ntos de los 900 enteros de tres d�gitos: 100, 101,
  102, \dots{}, 988, 999, est�n formados por una sucesi�n creciente de
  d�gitos? (Por ejemplo, 123 o 359, pero no 785 ni 788)

  \question �De cu�ntas maneras se pueden escoger tres n�meros
  naturales entre 1 y 30 (inclusive) de forma que su suma sea par?

\end{questions}

\end{document}

%%% Local Variables: 
%%% mode: latex
%%% TeX-master: t
%%% End: 
